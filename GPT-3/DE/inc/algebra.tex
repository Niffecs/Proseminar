\chapter{How does the Relations Algebra for Probabilistic Databases work ?}

The relations algebra for probabilistic databases is a mathematical system that defines a set of operations for manipulating and combining relations in a probabilistic database. A relation in a probabilistic database is a set of tuples that represent the possible combinations of values for a set of random variables. The relations algebra provides a set of operations that can be used to manipulate and combine these relations in order to perform probabilistic inference and answer queries about the data in the database. \\

The relations algebra for probabilistic databases is based on the relations algebra for conventional databases, but it includes additional operations and rules that are specific to probabilistic databases. Some of the key operations in the relations algebra for probabilistic databases include:\\

\begin{itemize}
	\item Projection: The projection operation selects a subset of the columns in a relation and discards the rest, creating a new relation with only the selected columns.
	
	\item Selection: The selection operation selects a subset of the rows in a relation based on a 
	
	\item Join: The join operation combines two relations by matching rows from each relation based on the values in a specified column or set of columns, creating a new relation with the combined rows.
	
	\item Union: The union operation combines two relations by taking the set of all rows that appear in either or both relations, creating a new relation with the combined rows.
	
	\item Intersection: The intersection operation combines two relations by taking the set of all rows that appear in both relations, creating a new relation with the combined rows.
	
	\item Difference: The difference operation combines two relations by taking the set of all rows that appear in the first relation but not in the second, creating a new relation with the combined rows.

\end{itemize}


These are just some of the key operations in the relations algebra for probabilistic databases. There are many other operations and rules that are part of the relations algebra, and they can be used to manipulate and combine relations in various ways in order to perform probabilistic inference and answer queries about the data in the database.