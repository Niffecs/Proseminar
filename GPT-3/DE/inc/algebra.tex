\chapter{Relationale Algebra}
Die Relationale Algebra ist ein formales System zur Beschreibung von Operationen auf relationalen Datenbanken. Sie wird verwendet, um Anfragen an eine relationale Datenbank zu formulieren und die Ergebnisse dieser Anfragen zu berechnen.


Die Relationale Algebra besteht aus einer Reihe von Operationen, die auf relationalen Datenbanken ausgeführt werden können, z.B. Selektion, Projektion, Vereinigung und Differenz. Diese Operationen können kombiniert werden, um komplexere Anfragen zu formulieren und die gewünschten Ergebnisse zu berechnen.

Die Relationale Algebra ist ein wichtiges Konzept in der Datenbankverarbeitung und wird häufig in der Praxis verwendet, um Anfragen an relationale Datenbanken zu formulieren und die Ergebnisse dieser Anfragen zu berechnen. Sie ist auch ein wichtiger Bestandteil der Theorie der relationalen Datenbanken und wird in der akademischen Forschung in diesem Bereich eingesetzt.






\section{Funktionalität}

Die Relationale Algebra ist ein formales System zur Beschreibung von Operationen auf relationalen Datenbanken. Sie wird verwendet, um Anfragen an eine relationale Datenbank zu formulieren und die Ergebnisse dieser Anfragen zu berechnen.
\vspace{1cm}

In einer probabilistischen Datenbank werden die Konzepte der Relationalen Algebra weitgehend gleich angewendet. Allerdings müssen die Operationen der Relationalen Algebra modifiziert werden, um die Wahrscheinlichkeiten von Ereignissen und Zuständen in der realen Welt zu berücksichtigen.
\vspace{1cm}

Zum Beispiel kann die Relationale Algebra in einer probabilistischen Datenbank verwendet werden, um Abfragen zu formulieren, die die Wahrscheinlichkeit bestimmter Ereignisse oder Zustände in der realen Welt berechnen. Dies kann mithilfe von Operationen wie Selektion, Projektion und Vereinigung erreicht werden.
\vspace{1cm}

Im Allgemeinen ermöglicht die Relationale Algebra in einer probabilistischen Datenbank die flexible Formulierung und Berechnung von Abfragen, die die Wahrscheinlichkeiten von Ereignissen und Zuständen in der realen Welt berücksichtigen.



\section{Unterschiede in der Struktur}
In der relationalen Algebra gibt es keine Unterschiede zwischen deterministischen und probabilistischen Datenbanken. Die relationale Algebra ist eine abstrakte Formelsprache, die zur Beschreibung von Operationen auf Relationen (also tabellarischen Datenstrukturen) verwendet wird. Sie kann sowohl auf deterministischen als auch auf probabilistischen Datenbanken angewendet werden.

Der Hauptunterschied zwischen deterministischen und probabilistischen Datenbanken besteht darin, wie die Daten in den Datenbanken gespeichert und verarbeitet werden. Deterministische Datenbanken verwenden deterministische Datenstrukturen und Algorithmen, die immer dasselbe Ergebnis liefern, wenn sie mit denselben Eingabedaten aufgerufen werden. Probabilistische Datenbanken hingegen verwenden probabilistische Datenstrukturen und Algorithmen, die unterschiedliche Ergebnisse liefern können, selbst wenn sie mit denselben Eingabedaten aufgerufen werden.

Obwohl die relationale Algebra in beiden Arten von Datenbanken verwendet werden kann, kann sie in probabilistischen Datenbanken dazu verwendet werden, probabilistische Abfragen zu formulieren und zu verarbeiten. In deterministischen Datenbanken hingegen werden die Operationen der relationalen Algebra hauptsächlich zur Verarbeitung deterministischer Abfragen verwendet.







