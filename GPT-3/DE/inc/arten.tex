\chapter{Datenbanken}

Es gibt verschiedene Arten von Datenbanken, die sich in ihrem Zweck, ihrer Größe, ihrer Struktur und ihren Funktionen unterscheiden. Zu den bekanntesten Arten von Datenbanken gehören relationale Datenbanken, NoSQL-Datenbanken und objektorientierte Datenbanken.

Relationale Datenbanken sind die am weitesten verbreiteten Datenbanken und basieren auf einem relationalen Datenmodell, das Daten in Tabellen speichert und über Beziehungen verknüpft. NoSQL-Datenbanken (Not Only SQL) sind nicht-relationale Datenbanken, die andere Datenmodelle als relationale Datenbanken verwenden, um Daten zu speichern und zu verarbeiten. Objektorientierte Datenbanken sind Datenbanken, die Daten als Objekte speichern und verarbeiten und sich somit an einem objektorientierten Programmierparadigma orientieren.

Es gibt auch spezialisierte Datenbanken wie probabilistische Datenbanken, die für die Verarbeitung und Analyse von probabilistischen Daten verwendet werden, und zeitbezogene Datenbanken, die für die Speicherung und Analyse von Daten in zeitlichen Abhängigkeiten verwendet werden.


\section{Arten von Datenbanken}
\subsection{Übersicht}
\begin{itemize}
\item	Relationale Datenbanken
\item	NoSQL-Datenbanken
\item 	Objektorientierte Datenbanken
\item	Probabilistische Datenbanken
\item	Zeitbezogene Datenbanken
\item	In-Memory-Datenbanken
\item	Dokumentenorientierte Datenbanken
\item	Graphdatenbanken
\item	Key-Value-Datenbanken
\item	Geodatenbanken
\item	Multimediale Datenbanken
\item	Zeitreihendatenbanken
\end{itemize}

\subsection{Relationale Datenbanken}
Relationale Datenbanken sind die am weitesten verbreiteten Arten von Datenbanken und basieren auf einem relationalen Datenmodell. In relationalen Datenbanken werden Daten in Tabellen gespeichert und über Beziehungen verknüpft. Jede Tabelle besteht aus Zeilen und Spalten, die als Tupel und Attribute bezeichnet werden. Tupel enthalten die Daten, während Attribute die Datenkategorien darstellen.


\subsection{NoSQL-Datenbanken}
NoSQL-Datenbanken (Not Only SQL) sind nicht-relationale Datenbanken, die andere Datenmodelle als relationale Datenbanken verwenden, um Daten zu speichern und zu verarbeiten. Im Gegensatz zu relationalen Datenbanken sind NoSQL-Datenbanken in der Regel weniger strukturiert und ermöglichen es, unterschiedliche Datenformate und -strukturen zu speichern.

NoSQL-Datenbanken werden oft in Szenarien verwendet, in denen hohe Skalierbarkeit, Flexibilität und Geschwindigkeit erforderlich sind, wie zum Beispiel bei der Verarbeitung von großen Mengen unstrukturierter Daten oder bei der Bereitstellung von real-time-Services im Internet. NoSQL-Datenbanken werden auch häufig in Cloud-Computing-Umgebungen verwendet, da sie es ermöglichen, Daten schnell und einfach auf mehrere Server zu verteilen.

Es gibt verschiedene Arten von NoSQL-Datenbanken, wie zum Beispiel Dokumentenorientierte Datenbanken, Key-Value-Datenbanken und Graphdatenbanken. Jede dieser Datenbanktypen hat ihre eigenen Stärken und Einsatzbereiche und wird für unterschiedliche Anwendungsfälle verwendet.





\subsection{Graphdatenbanken}
Graphdatenbanken sind eine spezielle Art von NoSQL-Datenbanken, die für die Speicherung und Verarbeitung von Daten in Form von Netzwerken oder Graphen verwendet werden. In Graphdatenbanken werden die Daten als Knoten (Nodes) und Beziehungen (Edges) dargestellt, wobei jeder Knoten einzelne Entitäten (z. B. Personen, Orte, Dinge) darstellt und die Beziehungen die Verbindungen zwischen den Entitäten beschreiben.

Graphdatenbanken eignen sich besonders gut für die Verarbeitung und Analyse von Daten, die sich gut in Beziehungen und Verbindungen darstellen lassen, wie zum Beispiel soziale Netzwerke, Verkehrsverbindungen oder Lieferketten. Sie bieten auch eine hohe Skalierbarkeit und Flexibilität, da sie es ermöglichen, Daten schnell und einfach zu verändern und zu erweitern.

Einige bekannte Beispiele für Graphdatenbanken sind Neo4j, JanusGraph und Titan. Sie werden in verschiedenen Bereichen wie dem Finanzwesen, der E-Commerce-Industrie und im Gesundheitswesen verwendet.


\subsection{Zeitreihendatenbanken}
Zeitreihendatenbanken sind spezielle Datenbanken, die für die Speicherung, Verarbeitung und Analyse von Daten in zeitlichen Abhängigkeiten entwickelt wurden. In Zeitreihendatenbanken werden die Daten als Zeitreihen gespeichert, die über einen bestimmten Zeitraum hinweg aufgezeichnet werden. Die Daten in einer Zeitreihe sind chronologisch geordnet und werden oft mit Metadaten wie Zeitstempel, Schlagwörtern und geografischen Informationen verknüpft.

Zeitreihendatenbanken eignen sich besonders gut für die Verarbeitung von Daten, die sich über einen bestimmten Zeitraum hinweg verändern, wie zum Beispiel Wetterdaten, Börsenkurse oder Verkehrsdaten. Sie bieten auch spezielle Funktionen wie die Möglichkeit, Daten in Echtzeit zu erfassen und zu analysieren, und die Möglichkeit, Daten auf verschiedenen Zeitskalen zu betrachten, was für viele Anwendungsfälle sehr nützlich ist.

Einige bekannte Beispiele für Zeitreihendatenbanken sind InfluxDB, KairosDB und OpenTSDB. Sie werden in verschiedenen Bereichen wie dem Finanzwesen, der Energieindustrie und im Gesundheitswesen verwendet.





