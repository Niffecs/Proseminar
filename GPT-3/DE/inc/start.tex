\chapter{Einleitung}
Probabilistische Datenbanken sind ein wichtiges Werkzeug in der heutigen digitalen Welt, in der die Verwaltung und Analyse großer Datenmengen eine zentrale Rolle bei der Lösung zahlreicher wissenschaftlicher und geschäftlicher Probleme spielt. Im Gegensatz zu konventionellen Datenbanken, die auf deterministischen Modellen basieren, ermöglichen probabilistische Datenbanken die Verwendung von probabilistischen Modellen, um die Unsicherheiten in den Daten zu berücksichtigen und die Genauigkeit von Anfragen und Vorhersagen zu verbessern.
\vspace{1cm}


Die Konzepte und Technologien probabilistischer Datenbanken haben in den letzten Jahren rasche Fortschritte gemacht und werden in verschiedenen Bereichen wie der Bioinformatik, der Finanzindustrie und der Naturwissenschaft eingesetzt. Zum Beispiel können probabilistische Datenbanken verwendet werden, um die Genexpression in biologischen Prozessen zu modellieren und zu analysieren, um die Risiken von Finanzportfolios besser zu verstehen und zu quantifizieren, oder um die Unsicherheiten in klimatischen Daten zu berücksichtigen und verbesserte Vorhersagen über den Klimawandel zu treffen.
\vspace{1cm}

In dieser Arbeit werden wir die grundlegenden Konzepte und Technologien probabilistischer Datenbanken untersuchen und ihre Anwendungen in verschiedenen Bereichen erläutern. Wir werden zunächst die Grundlagen probabilistischer Datenbanken und ihre Unterschiede zu konventionellen Datenbanken betrachten. Anschließend werden wir einige der wichtigsten Anwendungen probabilistischer Datenbanken in verschiedenen Bereichen untersuchen und die wichtigsten Herausforderungen und zukünftigen Entwicklungen in diesem Bereich diskutieren. Schließlich werden wir die Ergebnisse unserer Untersuchungen zusammenfassen und die Implikationen für die zukünftige Entwicklung probabilistischer Datenbanken darlegen.