\chapter{Literatur}
Es gibt eine reiche Literatur über probabilistische Datenbanken, die sich sowohl an Fachleute als auch an interessierte Laien richtet. Einige der wichtigsten Werke zum Thema probabilistische Datenbanken sind:

\begin{itemize}
	\item "Probabilistic Databases" von Luc De Raedt, Kristian Kersting, Sriraam Natarajan und Parag Singla: Dieses Buch bietet eine umfassende Einführung in die Konzepte und Techniken von probabilistischen Datenbanken und ihre Anwendungen.
	
	\item "Probabilistic Databases: From Theory to Practice" von Dan Suciu, Christopher Ré und Ralf Hartmut Güting: Dieses Buch bietet einen tiefen Einblick in die Theorie und Praxis von probabilistischen Datenbanken und ihre Verwendung in verschiedenen Anwendungsbereichen.
	
	\item "Probabilistic Databases: Techniques, Implementations and Applications" von Anish Das Sarma, Moritz Hardt, Daniel Hsu, Benjamin I. P. Rubinstein und Aleksandrs Slivkins: Dieses Buch bietet eine umfassende Übersicht über die verschiedenen Techniken, Implementierungen und Anwendungen von probabilistischen Datenbanken.
	
	
	\item "Probabilistic Relational Models for Knowledge Management" von Michael P. Wellman: Dieses Buch bietet eine detaillierte Einführung in die Verwendung von probabilistischen Relationalen Modellen für das Wissensmanagement und die Entscheidungsfindung.
	
	
\end{itemize}
Insgesamt gibt es eine reiche Auswahl an Literatur über probabilistische Datenbanken, die für Fachleute und interessierte Laien geeignet ist und sich mit verschiedenen Aspekten des Themas auseinandersetzt.



