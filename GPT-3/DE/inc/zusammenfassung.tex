\chapter{Zusammenfassung}

Probabilistische Datenbanken sind eine wichtige Innovation in der Welt der Datenverarbeitung, die es ermöglichen, unvollständige und unsichere Daten zu speichern und zu verarbeiten. Im Gegensatz zu deterministischen Datenbanken, die auf deterministischen Datenstrukturen und -algorithmen basieren, verwenden probabilistische Datenbanken probabilistische Datenstrukturen und -algorithmen, die es ermöglichen, die Unsicherheit und Unvollständigkeit von Daten direkt in der Datenbank zu berücksichtigen.

Eine wichtige Klasse von Datenstrukturen, die in probabilistischen Datenbanken verwendet werden, sind Graphen. Graphen sind eine allgemeine Datenstruktur, die es ermöglicht, Beziehungen zwischen verschiedenen Elementen zu modellieren und zu analysieren. In probabilistischen Datenbanken werden Graphen hauptsächlich zur Modellierung und Analyse von Beziehungen zwischen verschiedenen Entitäten (z. B. Personen, Orten, Dingen) verwendet.

Ein wichtiger Anwendungsbereich von probabilistischen Graphen in probabilistischen Datenbanken ist das Wissensmanagement und die Entscheidungsfindung. Durch die Verwendung von probabilistischen Graphen können wir Beziehungen zwischen verschiedenen Entitäten modellieren und analysieren, um wichtige Informationen und Einsichten zu gewinnen. Zum Beispiel können probabilistische Graphen verwendet werden, um die Wahrscheinlichkeit von bestimmten Ereignissen zu schätzen oder um die besten Entscheidungen in unsicheren oder unvollständigen Situationen zu treffen.

Ein weiterer wichtiger Anwendungsbereich von probabilistischen Graphen in probabilistischen Datenbanken ist die Verarbeitung von unstrukturierten oder semi-strukturierten Daten. Durch die Verwendung von probabilistischen Graphen können wir unvollständige oder inkonsistente Daten direkt in der Datenbank speichern und verarbeiten, ohne sie vorab zu bereinigen oder zu standardisieren. Dies ermöglicht es, schneller auf Veränderungen in den Daten zu reagieren und die Datenbank an sich verändernde Bedürfnisse anzupassen.



