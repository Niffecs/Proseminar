\chapter{Zusammenfassung}

Probabilistische Datenbanken sind eine relativ neue Klasse von Datenbanken, die auf probabilistischen Datenstrukturen und -algorithmen basieren. Im Gegensatz zu deterministischen Datenbanken ermöglichen probabilistische Datenbanken die Speicherung und Verarbeitung unvollständiger oder unsicherer Daten, was sie besonders für den Einsatz in modernen Big-Data-Umgebungen geeignet macht.

Ein wichtiges Konzept in probabilistischen Datenbanken sind Graphen. Graphen sind eine allgemeine Datenstruktur, die aus Knoten und Kanten besteht und dazu verwendet werden kann, verschiedene Arten von Beziehungen zwischen Daten zu modellieren. In probabilistischen Datenbanken werden Graphen verwendet, um die Unsicherheit und Unvollständigkeit von Daten zu repräsentieren und zu verarbeiten.

Es gibt verschiedene Arten von Graphen, die in probabilistischen Datenbanken verwendet werden können, je nachdem, welche Art von Beziehungen zwischen den Daten modelliert werden sollen. Zum Beispiel können Bayes-Netzwerke verwendet werden, um kausalen Zusammenhänge zwischen Daten zu modellieren, während Markov-Ketten dazu verwendet werden können, zeitliche Abhängigkeiten zwischen Daten zu modellieren.

Die Verwendung von Graphen in probabilistischen Datenbanken ermöglicht es, effektiv mit unvollständigen oder unsicheren Daten umzugehen und die Genauigkeit von Abfrageergebnissen zu verbessern. Es gibt verschiedene Algorithmen und Verfahren, die zur Verarbeitung von Graphen in probabilistischen Datenbanken verwendet werden können, wie zum Beispiel die Schätzverfahren von Pearl und D'Ambrosio und die Inferenzverfahren von Shafer und Pearl.

Insgesamt bieten probabilistische Datenbanken und ihre Graphen eine flexible und skalierbare Möglichkeit, mit unvollständigen und unsicheren Daten umzugehen und die Genauigkeit von Abfrageergebnissen zu verbessern. Sie werden in einer wachsenden Zahl von Anwendungsbereichen eingesetzt.


\cleardoublepage