\chapter{Vorteile}
Probabilistische Datenbanken bieten einige Vorteile gegenüber deterministischen Datenbanken, insbesondere in Bezug auf die Verarbeitung unsicheren oder unvollständigen Daten. Einige der wichtigsten Vorteile von probabilistischen Datenbanken sind:

\begin{enumerate}
	\item Flexibilität: Probabilistische Datenbanken ermöglichen es, unvollständige oder inkonsistente Daten zu speichern und zu verarbeiten, was in deterministischen Datenbanken nicht möglich ist. Dies ermöglicht es, schneller auf Veränderungen in den Daten zu reagieren und die Datenbank an sich verändernde Bedürfnisse anzupassen.
	
	\item Skalierbarkeit: Probabilistische Datenbanken können effektiv mit großen Mengen an unstrukturierten oder semi-strukturierten Daten umgehen, was sie besonders gut für den Einsatz in modernen Big-Data-Umgebungen geeignet macht.
	
	\item Genauigkeit: Probabilistische Datenbanken können unter Verwendung von probabilistischen Schätzverfahren und -algorithmen die Genauigkeit von Abfrageergebnissen verbessern, insbesondere bei der Verarbeitung von unvollständigen oder unsicheren Daten.
	
	\item Einfachheit: Die Verwendung von probabilistischen Datenbanken kann die Komplexität von Datenbankanwendungen reduzieren, da sie es ermöglichen, unsichere oder unvollständige Daten direkt in der Datenbank zu speichern und zu verarbeiten, anstatt sie vorab zu bereinigen oder zu standardisieren.
	
\end{enumerate}

Insgesamt bieten probabilistische Datenbanken eine flexible, skalierbare und genauere Möglichkeit, mit unsicheren oder unvollständigen Daten umzugehen, was sie besonders für den Einsatz in modernen Datenverarbeitungsumgebungen attraktiv macht.



