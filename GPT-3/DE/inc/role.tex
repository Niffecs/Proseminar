\chapter{What are the roles of graphs in probabilistic databases?}

Graphs play a key role in probabilistic databases by providing a way to represent and reason about the relationships between random variables. In a probabilistic database, a graph is a set of nodes, which represent random variables, and edges, which represent the relationships between the random variables. \\

The structure of the graph in a probabilistic database can provide important information about the dependencies and independencies between the random variables represented by the nodes. For example, if two nodes in the graph are connected by an edge, this indicates that the corresponding random variables are dependent, meaning that the value of one variable can affect the probability distribution of the other. On the other hand, if two nodes are not connected by an edge, this indicates that the corresponding random variables are independent, meaning that the value of one variable does not affect the probability distribution of the other. \\


In addition to representing dependencies and independencies between random variables, the structure of the graph in a probabilistic database can also be used to perform probabilistic inference, which is the process of using the known values of some random variables to reason about the likely values of other random variables. This can be useful for making predictions, making decisions, and answering queries about the data in the probabilistic database.\\


Overall, the use of graphs in probabilistic databases is an important aspect of representing and reasoning about uncertain or incomplete information. It allows probabilistic databases to represent the dependencies and independencies between random variables, and to perform probabilistic inference to make predictions, make decisions, and answer queries about the data in the database.\\


